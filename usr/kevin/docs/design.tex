\documentclass[11pt]{article}

% The following gives 21mm margin on all sides.

\setlength{\hoffset}{-1.0in}
\setlength{\voffset}{-1.0in}

\setlength{\headheight}{0.0cm}
\setlength{\headsep}{0.0cm}
\setlength{\topmargin}{3.1cm}
\setlength{\evensidemargin}{2.6cm}
\setlength{\oddsidemargin}{2.6cm}

\setlength{\textwidth}{15.8cm}
\setlength{\textheight}{22.9cm}

\setlength{\footskip}{0.7cm}
\setlength{\parindent}{0mm}
\setlength{\parskip}{\smallskipamount}

\usepackage{graphics}
\usepackage{graphicx}
\usepackage{epsf}
\usepackage{epic}

\begin{document}

%\frenchspacing

\section{Requirements}

\subsection{High Level Requirements}

In broad terms, the requirements for the FFS code are the following:
\begin{enumerate}
\item To provide a portable framework for FFS calculations
independent of the details of the simulation used.
\item To allow serial FFS calculations.
\item To allow trivially parallel FFS calculations (several FFS
``instances'' running concurrently), the statistics for which are
collated at the end.
\item To allow non-trivial parallelism, i.e., a single FFS ``instance''
decomposed in parallel, and/or using parallel simulation.
\item To provide at least direct and branched FFS implementations
\end{enumerate}

There are two specific simulation targets, with associated science:
\begin{enumerate}
\item FFS with LAMMPS, applied to enamtiomer formation.
\item FFS with lattice Boltzmann via \textit{Ludwig}, applied to
a liquid crystal problem.
\end{enumerate}


\section{Architecture}

\subsection{Overview}

We specify three main components of the system (see Figure~1).

\begin{figure}[h]
\end{figure}

\subsubsection{FFS core}

This is responsible for control of the overall FFS workflow. It should
accept input which specifies the details of algorithm to be used
(direct or branched, etc.), the number and duration of trials, and the
details of the number and poistion of the order parameter interfaces.
It must also allow control of the number of FFS instances, the number
of MPI tasks per instance, and the number of MPI tasks per simulation
trial.

The FFS core must be responsible for overall program control, for maintaining
an account of the trajectory or trajectories associated with the trials,  
issuing requests for new trials, control of random number seed information,
and so on.

The FFS core must output the final results in a digestable format, and
allow control over storage of the history of trajectories to an
appropriate level of detail.

\subsubsection{Controller}

This mediates the exchanges between the FFS core and the simulation
code so that the FFS core does not require any direct knowledge of
the simulation being undertaken.

The controller must handle the initiation trials as required by
the FFS algorithm, and return appropriate information on successful
and unsuccessful tirals.

It is imagined that the Controller will specify a fixed interface
by means of an abstract class. Any given simulation code will
require a concrete implementation of the class tailored to the
specific details of the simulation.

\subsubsection{Simulation}

This is reponsible for advancing the state in time, and computing
a relevant order parameter from the simulation state. It need have
no knowledge of the FFS procedure beyond what is dictated by the
Controller.

\begin{figure}[t]
\begin{center}
\input{./arch1.epic}
\end{center}
\caption{Architecture: Within \texttt{MPI\_COMM\_WORLD}, the driver
is expected to control one or more independent instances of the
FFS calculation (here two are illustrated). In each case, the FFS
library is responsible for the implementation of the appropriate
FFS algorithm, which will spawn a number of trials.
FFS issues requests for trial simulations to the Scheduler,
responsible for load balance. Requests are passed to the
Controller, which needs to be able to control the relevant
simulation, and return results to the FFS instance. For more than
one FFS instance, the driver
should collect the necessary aggregate statistics from the ensemble.}
\end{figure}


\section{Example: Gillespie Problem}




\section{Detailed Requirements}

The user will be concerned with specifying the details of the FFS
algorithm and related choices, and the particular parameters for
the simulation at hand. The user should not be concerned with the
details of scheduling, and the controller, which are the rightful
concerns of the developer. We set out here the requirements from
the user's view point.

\subsection{User Requirements}

\subsubsection{FFS Method}
\begin{itemize}
\item
MUST: Allow the user to specify a set of interfaces $\lambda_i$ explicitly
via an appropriate input mechnaism.
\item
SHOULD: Allow the user to specify a set of evenly spaced interfaces
implicitly, by specifying the number of interfaces, and $\lambda_{min}$
and $\lambda_{max}$.
\item
COULD: Allow the user some automatic choice, or dynamic repositioning, of
interfaces via appropriate algorithm given $\lambda_{}$ and $\lambda_{max}$.
\item
SHOULD: Allow the user to specify an intial 'relaxation' run period, or
start immediately from a pre-specified initial state.
\item
MUST: Allow the user to specify whether direct or branched FFS is required.
\item
MUST: Allow the user to specify the number of trials involved.
\item
MUST: Allow the user to specify a maximum simulation time for any trial
(which will include a description of what time means for the simulation,
e.g., integer number of steps, or floating point time interval in given
units).
\end{itemize}

\subsubsection{Parallelism}
\begin{itemize}
\item
MUST: Allow the user the choice of running in FFS serial, or in parallel.
\item
MUST: Allow the user to specify the number of FFF 'instances' which will
run in parallel, and how many MPI tasks each instance will have.
\item
MUST: Allow the user to specify if the simulation will run in parallel
or not (does it take an MPI communicator?), and if so, on how many
MPI tasks.
\item
COULD: Provide user control of whether dynamic load balance is attempted
or not.
\end{itemize}

\subsubsection{Miscellaneous}

\begin{itemize}
\item
MUST: The user must be able to extract appropriate statistical information
in aggregate form.
\item
SHOULD: Allow some level of control of the detail of statistical output.
\item
SHOULD: Allow the user to split a single FFS calculation into successive
jobs. That is, the user should be able to 'restart' from a saved state
if the total run time is expected to be large.
\end{itemize}

\subsubsection{Simulation}


\subsection{Developer Requirements}


\subsection{Controller}




\end{document}
