%%%%%%%%%%%%%%%%%%%%%%%%%%%%%%%%%%%%%%%%%%%%%%%%%%%%%%%%%%%%%%%%%%%%%%%%%%%%%%
%%
%%  init2a.tex
%%
%%  To illustrate serial and parallel initial trajectories.
%%  (2a) Single serial trajectory
%%
%%  Convert to gif and reduce to 66% for final gif for web.
%%
%%%%%%%%%%%%%%%%%%%%%%%%%%%%%%%%%%%%%%%%%%%%%%%%%%%%%%%%%%%%%%%%%%%%%%%%%%%%%%

\documentclass{article}
\usepackage{pgf}
\usepackage{tikz}
\usepackage{verbatim}
\usepackage[active,tightpage]{preview}
\PreviewEnvironment{tikzpicture}
\setlength\PreviewBorder{5pt}%
\usetikzlibrary{arrows, shapes, petri}

\begin{document}
\pagestyle{empty}

\def\StateA{StateA}
\def\StateONE{StateONE}
\def\StateTWO{StateTWO}
\def\Sref{Sref}
\def\LambdaONE{LambdaONE}
\def\LambdaTWO{LambdaTWO}
\def\LambdaDOT{LambdaDOT}


\begin{tikzpicture}[every node/.style={font=\normalsize,
  minimum height=0.5cm,minimum width=0.5cm},]

% Here's the grid

\node [matrix, column sep = 1.2cm, row sep = 0.3cm] (matrix) at (0,0) {

  \node(0,0) (\StateA 1) {}; & & & \node(0,0) (\LambdaONE 1) {}; &
   & \node(0,0) (\LambdaTWO 1) {}; & \node(0,0) (\LambdaDOT 1) {}; \\

  & & & \node(0,0) (State3) {}; & &  \\

  & \node(0,0) (t1 left) {};  & & \node(0,0) (t1 l1) {}; & & & \\

  \node(0,0) (\StateA 2) {}; & & & \node(0,0) (\StateTWO) {}; &
   \node(0,0) (\StateTWO l2) {}; & & \\

  \node(0,0) (t2 left) {}; & & \node(0,0) (t2 mid) {};
  & \node(0,0) (t2 l1) {}; & & & \\

  & \node(0,0) (\Sref) {}; & &\node(0,0) (\StateONE) {}; & & &  \\

  & & & \node(0,0) (\LambdaONE 3) {}; & & \node(0,0) (\LambdaTWO 3) {};
  & \node(0,0) (\LambdaDOT 3) {}; \\
};


% Horizontal progress
%\draw [-latex,dashed] 
%  (t3 left) -- (t3 right.east) node[right] {\it Progress};

% Vertical Interfaces

\draw
  (\LambdaONE 1.north) -- (\LambdaONE 3.south) node[below] {$\lambda_1$}
  (\LambdaTWO 1.north) -- (\LambdaTWO 3.south) node[below] {$\lambda_2$}
  (\LambdaDOT 1.north) -- (\LambdaDOT 3.south) node[below] {$\ldots$};
%  (\StateB 1.north) -- (\StateB 3.south) node[below] {$\lambda_n$};

% Blocks for A and B states

\filldraw[fill=blue!20]
  (\StateA 1.north) rectangle (\LambdaONE 3.south);
%  (\StateB 1.north) rectangle (\StateR.south);

\fill
  (\StateA 1) node[right] {STATE A};
%  (\StateB 1) node[right] {STATE B};

% States
\draw
  (\Sref) node[draw, circle, fill=green!20] {$S_{ref}$}
  (\StateONE) node[draw,circle,fill=green!20] {$S_1$}
  (\StateTWO) node[draw,circle,fill=green!20] {$S_2$}
  (State3) node[draw,circle,fill=green!20] {$S_3$};

% Trjectory: Sref -> S1
\path[->, thick, >=stealth', shorten <=6pt, shorten >=3pt]
(\Sref) edge [out=-60, in = 180] node[auto] {$t_1$} (\StateONE.west);
% S1 -> S2 via lamnda 2
\path[-, thick, shorten <=4pt]
(\StateONE.east) edge [out = 0, in = 180] (\LambdaTWO 3.center);
\path[->, thick, >=stealth']
(\LambdaTWO 3.center) edge [out = 0, in = 0] (t2 l1.center);
\path[-,thick]
(t2 l1.center) edge [out = 180, in = -45] (t2 mid.center);
\path[->, thick, >=stealth', shorten >=3pt]
(t2 mid.center) edge [out = 135, in = 180] (\StateTWO.west);
% S2 -> S3
\path[-, thick, shorten <=4pt]
(\StateTWO.east) edge [out = 0, in = -135] (\StateTWO l2.center);
\path[->, thick, >=stealth']
(\StateTWO l2.center) edge [out = 45, in = 0] (t1 l1.center);
\path[-, thick]
(t1 l1.center) edge [out = 180, in = -135] (t1 left.east);
\path[->, thick, >=stealth', shorten >=3pt]
(t1 left.east) edge [out = 45, in = 135] (State3.north west);
\path[->, thick, >=stealth', shorten >=20pt]
(State3.south east) edge [out = -45, in = -135] (\LambdaTWO 1.west);

\node (Label) [above of=\StateA 1] {};
\node (LabelText) [right of=Label] {(a) Single trajectory};

\end{tikzpicture}
\end{document}
