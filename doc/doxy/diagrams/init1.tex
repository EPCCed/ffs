%%%%%%%%%%%%%%%%%%%%%%%%%%%%%%%%%%%%%%%%%%%%%%%%%%%%%%%%%%%%%%%%%%%%%%%%%%%%%%
%%
%%  init1.tex
%%
%%  Generic picture for generation of initial states.
%%
%%  Convert to gif and reduce to 9cm in height for final gif for web.
%%
%%%%%%%%%%%%%%%%%%%%%%%%%%%%%%%%%%%%%%%%%%%%%%%%%%%%%%%%%%%%%%%%%%%%%%%%%%%%%%

\documentclass{article}
\usepackage{pgf}
\usepackage{tikz}
\usepackage{verbatim}
\usepackage[active,tightpage]{preview}
\PreviewEnvironment{tikzpicture}
\setlength\PreviewBorder{5pt}%
\usetikzlibrary{arrows, shapes, petri}

\begin{document}
\pagestyle{empty}

\def\StateA{StateA}
\def\StateB{StateB}
\def\StateONE{StateONE}
\def\StateTWO{StateTWO}
\def\StateTHREE{StateTHREE}
\def\Sref{Sref}
\def\LambdaONE{LambdaONE}
\def\LambdaTWO{LambdaTWO}
\def\LambdaDOT{LambdaDOT}
\def\StateR{StateR}


\begin{tikzpicture}[every node/.style={font=\normalsize,
  minimum height=0.5cm,minimum width=0.5cm},]

% Here's the grid

\node [matrix, column sep = 1.2cm, row sep = 0.3cm] (matrix) at (0,0) {

  \node(0,0) (\StateA 1) {}; & & & \node(0,0) (\LambdaONE 1) {};
   & \node(0,0) (\LambdaTWO 1) {}; & \node(0,0) (\LambdaDOT 1) {};
  & \node(0,0) (\StateB 1) {}; & & \\

  & & & & & & & \\

  \node(0,0) (t1 left) {}; & & & \node(0,0) (\StateONE) {}; & & & &
  \node(0,0) (t1 right) {}; &\\

  \node(0,0) (\StateA 2) {}; & \node(0,0) (\Sref 2) {}; & &
    & & & & & \\

  \node(0,0) (t2 left) {}; & & & \node(0,0) (\StateTWO) {}; & & & &
  \node(0,0) (t2 right) {}; & \\

  & & \node(0,0) (\StateTHREE) {}; & & & & & & \\

  & & & \node(0,0) (\LambdaONE 3) {}; & \node(0,0) (\LambdaTWO 3) {};
  & \node(0,0) (\LambdaDOT 3) {}; 
    & \node(0,0) (\StateB 3) {}; & & \node(0,0) (\StateR) {}; \\

  \node(0,0) (t3 left) {}; & & & & & & \node(0,0) (t3 right) {}; & &\\
};


% Horizontal progress
\draw [-latex,dashed] 
  (t3 left) -- (t3 right.east) node[right] {\it Progress};

% Vertical Interfaces

\draw
  (\LambdaONE 1.north) -- (\LambdaONE 3.south) node[below] {$\lambda_1$}
  (\LambdaTWO 1.north) -- (\LambdaTWO 3.south) node[below] {$\lambda_2$}
  (\LambdaDOT 1.north) -- (\LambdaDOT 3.south) node[below] {$\ldots$}
  (\StateB 1.north) -- (\StateB 3.south) node[below] {$\lambda_n$};

% Blocks for A and B states

\filldraw[fill=blue!20]
  (\StateA 1.north) rectangle (\LambdaONE 3.south)
  (\StateB 1.north) rectangle (\StateR.south);

\fill
  (\StateA 1) node[right] {STATE A}
  (\StateB 1) node[right] {STATE B};

% States
\draw
  (\Sref 2) node[draw, circle, fill=green!20] {$S_{ref}$}
  (\StateONE) node[draw,circle,fill=green!20] {$S_1$}
  (\StateTWO) node[draw,circle,fill=green!20] {$S_2$}
  (\StateTHREE) node[draw,circle,fill=red!20] {$S_3$};

% Trjectories
\path[->, thick, >=stealth', shorten <=5pt, shorten >=3pt]
(\Sref 2) edge [out=45, in=135] node[auto] {$t_1$} (\StateONE.north west);

\path[->, thick, >=stealth', shorten <=6pt, shorten >=3pt]
(\Sref 2) edge [out=-60, in=135] node[auto] {$t_2$} (\StateTWO.north west);

\path[->, thick, >=stealth', shorten <=8pt, shorten >=3pt]
(\Sref 2) edge [out=-90, in=-180] node[auto] {$t_3$} (\StateTHREE.west);

\end{tikzpicture}
\end{document}
