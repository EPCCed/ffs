%%%%%%%%%%%%%%%%%%%%%%%%%%%%%%%%%%%%%%%%%%%%%%%%%%%%%%%%%%%%%%%%%%%%%%%%%%%%%%
%%
%%  library2.tex
%%
%%  Block diagram for library / interface / simulation.
%%
%%  Convert to gif and reduce to 66% for final gif for web.
%%
%%%%%%%%%%%%%%%%%%%%%%%%%%%%%%%%%%%%%%%%%%%%%%%%%%%%%%%%%%%%%%%%%%%%%%%%%%%%%%

\documentclass{article}
\usepackage{pgf}
\usepackage{tikz}
\usepackage{verbatim}
\usepackage[active,tightpage]{preview}
\PreviewEnvironment{tikzpicture}

\usetikzlibrary{arrows, shapes, positioning, decorations.markings}

\tikzstyle{block}
= [draw, rectangle, minimum height = 0.5cm, minimum width = 3.2cm]

% From TeXample.net by Dominik Haumann
\tikzstyle{vecArrow} = [thick, decoration={markings,mark=at position
   1 with {\arrow[semithick]{open triangle 60}}},
   double distance=1.4pt, shorten >= 5.5pt,
   preaction = {decorate},
   postaction = {draw,line width=1.4pt, white,shorten >= 4.5pt}]

\begin{document}
\pagestyle{empty}

\begin{tikzpicture}[every node/.style={font=\scriptsize}]

\node[block, fill=blue!10] (N1) {FFS LIBRARY};

\node[block, below = 0.4 cm of N1, fill = yellow!10] (N2) {INTERFACE LAYER};

\node[block, below = 0.4 cm of N2, fill = red!10] (N3) {SIMULATION};


\node[minimum width = 1.0cm] (a) at (N1.south) {};
\node[minimum width = 1.0cm] (b) at (N2.north) {};
\draw[vecArrow] (a.west) to (b.west);
\draw[vecArrow] (b.east) to (a.east);


\node[minimum width = 1.0cm] (a) at (N2.south) {};
\node[minimum width = 1.0cm] (b) at (N3.north) {};
\draw[vecArrow] (a.west) to (b.west);
\draw[vecArrow] (b.east) to (a.east);

\end{tikzpicture}
\end{document}
