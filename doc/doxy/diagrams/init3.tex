%%%%%%%%%%%%%%%%%%%%%%%%%%%%%%%%%%%%%%%%%%%%%%%%%%%%%%%%%%%%%%%%%%%%%%%%%%%%%%
%%
%%  init3.tex
%%
%%  To illustrate serial and parallel initial trajectories.
%%  Parallel
%%
%%  Convert to gif and reduce to 66% for final gif for web.
%%
%%%%%%%%%%%%%%%%%%%%%%%%%%%%%%%%%%%%%%%%%%%%%%%%%%%%%%%%%%%%%%%%%%%%%%%%%%%%%%

\documentclass{article}
\usepackage{pgf}
\usepackage{tikz}
\usepackage{verbatim}
\usepackage[active,tightpage]{preview}
\PreviewEnvironment{tikzpicture}
\setlength\PreviewBorder{5pt}%
\usetikzlibrary{arrows, shapes, petri}

\begin{document}
\pagestyle{empty}


\begin{tikzpicture}[every node/.style={font=\normalsize,
  minimum height=0.5cm,minimum width=0.5cm},]

% Here's the grid

\node [matrix, column sep = 1.2cm, row sep = 0.2cm] (matrix) at (0,0) {

  \node(0,0) (StateA) {}; & & & \node(0,0) (Top-l1) {}; &
   & \node(0,0) (Top-l2) {}; & \node(0,0) (Top-ldot) {}; \\

  & & & \node(0,0) (State1) {}; & &  \\
  & & \node(0,0) (t1 left) {}; & & \node(0,0) (t1-l1) {};  & & \\

  & & & \node(0,0) (State2) {}; & & \node(0,0) (State2-l2) {}; & \\
  & \node(0,0) (Sref) {}; & & & & & \\

  & & &\node(0,0) (State3) {}; & & &  \\
  & & & & & \node(0,0) (t34-l2) {}; & \\
  & & & \node(0,0) (State4) {}; & \node(0,0) (t4-mid) {}; & & \\

  & & & \node(0,0) (Bottom-l1) {}; & & \node(0,0) (Bottom-l2) {};
  & \node(0,0) (Bottom-ldot) {}; \\
};


% Vertical Interfaces

\draw
  (Top-l1.north) -- (Bottom-l1.south) node[below] {$\lambda_1$}
  (Top-l2.north) -- (Bottom-l2.south) node[below] {$\lambda_2$}
  (Top-ldot.north) -- (Bottom-ldot.south) node[below] {$\ldots$};

% Blocks for A and B states

\filldraw[fill=blue!20]
  (StateA.north) rectangle (Bottom-l1.south);

\fill
  (StateA) node[right] {STATE A};

% States
\draw
  (Sref) node[draw, circle, fill=green!20] {$S_{ref}$}
  (State1) node[draw,circle,fill=green!20] {$S_1$}
  (State2) node[draw,circle,fill=green!40] {$S_2$}
  (State3) node[draw,circle,fill=green!60] {$S_3$}
  (State4) node[draw,circle,fill=green!80] {$S_4$};

% Trajectory t1: Sref -> S1 -> ...

\path[->, thick, >=stealth', shorten <=6pt, shorten >=3pt]
(Sref) edge [out=60, in = 180] node[auto] {$t_1$} (State1.west);
\path[->, thick, >=stealth', shorten <=4pt, shorten >=10pt]
(State1.east) edge [out=0, in = 180] (Top-l2.south);

% Trajectory t2: Sref -> S2 -> ...

\path[->, thick, >=stealth', shorten <=6pt, shorten >=3pt]
(Sref) edge [out=30, in = -170] node[auto] {$t_2$} (State2.west);
\path[->, thick, >=stealth', shorten <=4pt, shorten >=3pt]
(State2.east) edge [out=10, in = 170] (t1-l1.center);

% Trajectory t3: Sref -> S3 -> ...

\path[->, thick, >=stealth', shorten <=6pt, shorten >=3pt]
(Sref) edge [out=-45, in = 170] node[auto] {$t_3$} (State3.west);
\path[->, thick, >=stealth', shorten <=4pt, shorten >=3pt]
(State3.east) edge [out=-10, in = 100] (t34-l2.west);



% Trajectory t4: Sref -> S4...

\path[->, thick, >=stealth', shorten <=8pt, shorten >=2pt]
(Sref.south) edge [out=-90, in = -135] node[auto] {$t_4$} (State4.south west);
\path[->, thick, >=stealth', shorten <=2pt, shorten >=2pt]
(State4.north east) edge [out=45, in = -135]  (Bottom-l2.west);



% LABEL
\node (Label) [above of=StateA] {};
\node (LabelText) [right of=Label] {(c) Parallel trajectories};

\end{tikzpicture}
\end{document}
