%%%%%%%%%%%%%%%%%%%%%%%%%%%%%%%%%%%%%%%%%%%%%%%%%%%%%%%%%%%%%%%%%%%%%%%%%%%%%%
%%
%%  init1.tex
%%
%%  To illustrate serial and parallel initial trajectories.
%%  (b) Parallel
%%
%%  Convert to gif and reduce to 66% for final gif for web.
%%
%%%%%%%%%%%%%%%%%%%%%%%%%%%%%%%%%%%%%%%%%%%%%%%%%%%%%%%%%%%%%%%%%%%%%%%%%%%%%%

\documentclass{article}
\usepackage{pgf}
\usepackage{tikz}
\usepackage{verbatim}
\usepackage[active,tightpage]{preview}
\PreviewEnvironment{tikzpicture}
\setlength\PreviewBorder{5pt}%
\usetikzlibrary{arrows, shapes, petri}

\begin{document}
\pagestyle{empty}


\begin{tikzpicture}[every node/.style={font=\normalsize,
  minimum height=0.5cm,minimum width=0.5cm},]

% Here's the grid

\node [matrix, column sep = 1.2cm, row sep = 0.2cm] (matrix) at (0,0) {

  \node(0,0) (StateA) {}; & & & \node(0,0) (Top-l1) {}; &
   & \node(0,0) (Top-lambda2) {}; & \node(0,0) (Top-ldot) {}; \\

  & & & \node(0,0) (State1) {}; & &  \\
  & & \node(0,0) (t1 left) {}; & \node(0,0) (t1 l1) {}; & & & \\
  & & & \node(0,0) (State2) {}; & & \node(0,0) (State2-l2) {}; & \\
  \node(0,0) (t2 left) {}; & & \node(0,0) (t2 mid) {};
  & \node(0,0) (t2 l1) {}; & & \node(0,0) (t2-l2) {}; & \\
  & & &\node(0,0) (State3) {}; & & &  \\
  & \node(0,0) (Sref) {}; & & & & & \\
  & & & \node(0,0) (State4) {}; & \node(0,0) (t4-mid) {}; & & \\
  & & & \node(0,0) (t5-l1) {}; & & & \\
  \node(0,0) (t5-left) {}; & & & \node(0,0) (State5) {}; & & & \\
  & & & \node(0,0) (t6-l1) {}; & & & \\
  & & \node(0,0) (t7-mid) {}; & \node(0,0) (State6) {}; & & & \\

  & & & \node(0,0) (Bottom-l1) {}; & & \node(0,0) (Bottom-lambda2) {};
  & \node(0,0) (Bottom-ldot) {}; \\
};


% Horizontal progress
%\draw [-latex,dashed] 
%  (t3 left) -- (t3 right.east) node[right] {\it Progress};

% Vertical Interfaces

\draw
  (Top-l1.north) -- (Bottom-l1.south) node[below] {$\lambda_1$}
  (Top-lambda2.north) -- (Bottom-lambda2.south) node[below] {$\lambda_2$}
  (Top-ldot.north) -- (Bottom-ldot.south) node[below] {$\ldots$};

% Blocks for A and B states

\filldraw[fill=blue!20]
  (StateA.north) rectangle (Bottom-l1.south);

\fill
  (StateA) node[right] {STATE A};

% States
\draw
  (Sref) node[draw, circle, fill=green!20] {$S_{ref}$}
  (State1) node[draw,circle,fill=green!20] {$S_1$}
  (State2) node[draw,circle,fill=green!20] {$S_2$}
  (State3) node[draw,circle,fill=green!20] {$S_3$}
  (State4) node[draw,circle,fill=green!40] {$S_4$}
  (State5) node[draw,circle,fill=green!40] {$S_5$}
  (State6) node[draw,circle,fill=green!40] {$S_6$};

% Trajectory t1: Sref -> S1 -> ...

\path[->, thick, >=stealth', shorten <=4pt, shorten >=3pt]
(Sref) edge [out=60, in = 180] node[auto] {$t_1$} (State1.west);
% S1 -> S2
\path[-, thick, shorten <=4pt]
(State1.east) edge [out = 0, in = 180] (Top-lambda2.south);
\path[->, thick, >=stealth']
(Top-lambda2.south) edge [out = 0, in = 0] (t1 l1.center);
\path[-,thick]
(t1 l1.center) edge [out = 180, in = 45] (t1 left.south east);
\path[->, thick, >=stealth', shorten >=3pt]
(t1 left.south east) edge [out = -135, in = 180] (State2.west);
% S2 -> S3
\path[-, thick, shorten <=4pt]
(State2.east) edge [out = 0, in = 150] (State2-l2.west);
\path[->, thick, >=stealth']
(State2-l2.west) edge [out = -30, in = 0] (t2 l1.center);
\path[-, thick]
(t2 l1.center) edge [out = 180, in = -135] (t2 mid.east);
\path[->, thick, >=stealth', shorten >=3pt]
(t2 mid.east) edge [out = 45, in = 135] (State3.north west);
% S3 -> ...
\path[->, thick, >=stealth', shorten >=3pt]
(State3.south east) edge [out = -45, in = -135] (t2-l2.west);


% Trajectory t2: Sref -> S4...

\path[->, thick, >=stealth', shorten <=4pt, shorten >=3pt]
(Sref.south) edge [out=-90, in = 135] node[auto] {$t_2$} (State4.north west);

% S4 -> S5
\path[-, thick]
(State4.south east) edge [out=-45, in = 135] (t4-mid.center);
\path[->, thick, >=stealth']
(t4-mid.center) edge [out=-45, in = 0] (t5-l1.center);
\path[-, thick]
(t5-l1.center) edge [out=180, in = 90] (t5-left.east);
\path[->, thick, >=stealth', shorten >=3pt]
(t5-left.east) edge [out=-90, in = 180] (State5.west);

% S5 -> S6
\path[-, thick]
(State5.east) edge [out=0, in = 135] (Bottom-ldot.east);
\path[->, thick, >=stealth']
(Bottom-ldot.east) edge [out=-45, in = 0] (t6-l1.center);
\path[-, thick]
(t6-l1.center) edge [out=180, in = 90] (t7-mid.east);
\path[->, thick, >=stealth', shorten >=2pt]
(t7-mid.east) edge [out=-90, in = -135] (State6.south west);

% S6 -> ..
\path[->, thick, >=stealth', shorten >=10pt]
(State6.north east) edge [out=45, in = 180] (Bottom-lambda2.south west);


% LABEL
\node (Label) [above of=StateA] {};
\node (LabelText) [right of=Label] {(b) Parallel trajectories};

\end{tikzpicture}
\end{document}
