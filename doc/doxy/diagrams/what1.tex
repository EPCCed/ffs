%%%%%%%%%%%%%%%%%%%%%%%%%%%%%%%%%%%%%%%%%%%%%%%%%%%%%%%%%%%%%%%%%%%%%%%%%%%%%%
%%
%%  what1.tex
%%
%%  Illustration of what FFS does. (a) Note this is a subset of
%%  what2.tex.
%%
%%  Convert to gif and reduce to 66% for final gif for web.
%%
%%%%%%%%%%%%%%%%%%%%%%%%%%%%%%%%%%%%%%%%%%%%%%%%%%%%%%%%%%%%%%%%%%%%%%%%%%%%%%

\documentclass{article}
\usepackage{pgf}
\usepackage{tikz}
\usepackage{verbatim}
\usepackage[active,tightpage]{preview}
\PreviewEnvironment{tikzpicture}
\setlength\PreviewBorder{5pt}%
\usetikzlibrary{arrows, shapes, petri}

\begin{document}
\pagestyle{empty}


\begin{tikzpicture}[every node/.style={font=\normalsize,
  minimum height=0.5cm,minimum width=0.5cm},]

% Here's the grid

\node [matrix, column sep = 1.2cm, row sep = 0.2cm] (matrix) at (0,0) {

  \node(0,0) (StateA) {}; & \node(0,0) (Top-l1) {}; &
   & \node(0,0) (Top-l2) {};  & & \node(0,0) (Top-l3) {}; \\

  \node(0,0) (t4-left) {}; & \node(0,0) (State1) {}; &
  & \node(0,0) (State2) {}; & & \\

  &  \node(0,0) (t5-l1) {}; & \node(0,0) (t5-l12) {}; & & & \\

  \node(0,0) (Bottom-left) {}; &  \node(0,0) (Bottom-l1) {}; & &
  \node(0,0) (Bottom-l2) {}; & & \node(0,0) (Bottom-l3) {}; \\
};


% Vertical Interfaces

\draw
  (Top-l1.north) -- (Bottom-l1.south) node[below] {$\lambda_1$}
  (Top-l2.north) -- (Bottom-l2.south) node[below] {$\lambda_2$};

% States
\draw
  (State2) node[draw,circle, minimum height = .8cm, fill=green!20] {$S_2$}
  (State1) node[draw,circle, minimum height = .8cm, fill=green!20] {$S_1$};


% S4 -> ...
\path[->, thick, >=stealth', shorten <=12pt, shorten >=4pt]
(t4-left.east) edge [out = 0, in = 170] (State1.west);

\path[->, thick, >=stealth', shorten <=4pt, shorten >=4pt]
(State1.east) edge [out = 0, in = -135] (State2.west);

\path[-, shorten <=4pt]
(State1.east) edge [out = -10, in = 135] (t5-l12.center);
\path[->, >=stealth']
(t5-l12.center) edge [out = -45, in = 20] (Bottom-left.east);


% LABEL
\node (Label) [above of=StateA] {};
\node (LabelText) [right of=Label] {(a) States and trajectories};

\end{tikzpicture}
\end{document}
